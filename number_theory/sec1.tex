\section{Numbers and Sets}\label{sec:numbers-and-sets}

\begin{itemize}
    \item natural numbers
    \item divides ------ $\exists k$ st $b = ka$
    \item factor
    \item divisor
    \item divisible
    \item prime ------ only factor are 1 and $n$
    \item composite
    \item prime counting function $\pi(x)$ ------ $\#$ primes $\leq x$
\end{itemize}

\begin{lemma}
    $n > 1$, then $n$ has prime factor
\end{lemma}

\begin{thm}
    $\exists$ infinitely many primes
\end{thm}

\begin{itemize}
    \item highest common factor / greatest common divisor
    \item coprime / relatively prime
    \item Euclid's algorithm
\end{itemize}

\begin{prop}
    Euclid's algorithm works
\end{prop}

\begin{thm}[Bezout]
    $a,b,c \in \N$, then $\exists m, n$ st $am + bn = c$ $\iff$ $(a,b) \mid c$
\end{thm}

\begin{prop}
    $p$ prime, $p \mid ab $, then $p \mid a$ or $p \mid b$
\end{prop}
\begin{pf}
    assume $p \nmid a$, then Bezout
\end{pf}

\begin{thm}[Fundamental Theorem of Arithmetic]
    $n \in \N$, then $n$ can be factorised as product of primes uniquely (up to reordering)
\end{thm}
\begin{pf}
    Existence: induction\\
    Uniqueness: $p_1 \mid q_1\cdots q_k$
\end{pf}

\begin{itemize}
    \item congruent to $b$ modulo $n$ ------ $n \mid a - b$
\end{itemize}

\begin{lemma}
    $n  >1$, $(a, n) = 1$, then $\exists m$ st $am \equiv 1$ (multiplicative inverse mod $n$)
\end{lemma}
\begin{pf}
    Bezout
\end{pf}

\begin{itemize}
    \item unit ------ invertible elements
    \item multiplicative group $(\Z/n\Z)^\times$ or $(\Z/n\Z)^*$ ------ group of unit
    \item Euler totient function $\phi(n) = \left|(\Z/n\Z)^\times  \right|$
\end{itemize}

\begin{fact}
    $\phi(p) = p - 1$
\end{fact}

\begin{thm}[Fermat-Euler]
    $n > 1$, $(a , n ) = 1$, then $a^\phi(n) \equiv 1 \pmod n$
\end{thm}
\begin{pf}
    Langrange's
\end{pf}

\begin{cor}[Fermat's Little Theorem]
    $a^{p-1} \equiv 1 \pmod{p}$
\end{cor}

\begin{thm}[Chinese remainder theorem]
    $m_1, m_2 > 1$, $(m_1, m_2) = 1$, $a_1, a_2 \in \Z$, then $\exists n$ st
    $\begin{cases}
         n \equiv a_1 \pmod{m_1}\\
         n \equiv a_2 \pmod{m_2}
    \end{cases}$, unique up to modulo $m_{1}m_2$
\end{thm}

\begin{fact}
    extend to more congruences as long as pairwise coprime
\end{fact}

\begin{fact}
    $\Z/n\Z \cong \Z/p_1^{\alpha_1}\Z \times \cdots \times \Z/p_k^{\alpha_k}\Z$
\end{fact}

\begin{cor}
    In addition, $(a_1, m_1) = 1, (a_2, m_2) = 1$, then $(n, m_1 m_2) = 1$
\end{cor}

\begin{fact}
    $(\Z/n\Z)^\times \cong (\Z/p_1^{\alpha_1}\Z)^\times \times \cdots \times (\Z/p_k^{\alpha_k}\Z)^\times$
\end{fact}

\begin{itemize}
    \item multiplicative ------ $f(mn) = f(m)f(n)$ whenever $m, n$ coprime
    \item totally multiplicative ------ $f(mn) = f(m)f(n)$ for all $m, n$
\end{itemize}

\begin{cor}
    $\phi$ Euler function multiplicative
\end{cor}
\begin{pf}
    $(\Z/m_1 m_2\Z)^\times = (\Z/m_1\Z)^\times \times (\Z/m_2\Z)^\times$
\end{pf}

\begin{lemma}
    $p$ prime, $k \in \N$, then $\phi(p^k) = p^{k-1}(p - 1)$
\end{lemma}
\begin{pf}
    direct counting $p^k - p^{k-1}$
\end{pf}

\begin{itemize}
    $\sum_{d \mid n} \phi(d)$
\end{itemize}

\begin{lemma}
    $n \in \N$, then $\sum_{d \mid n} \phi(d) = n$
\end{lemma}
\begin{pf}
    prove multiplicity, then work on $p^k$
\end{pf}

\begin{cor}
    $f$ multiplicative $\Rightarrow$ $\sum_{d \mid n} f(d)$ multiplicative
\end{cor}

\begin{itemize}
    \item $d(n) = \tau(n) = \sum_{d \mid n} 1 = $ \# divisors
    \item $\sigma(n) = \sum_{d \mid n} d = $ sum of divisors
\end{itemize}

\begin{thm}[Lagrange Theorem]
    $p$ prime, $f(x) = a_n x^n + \dots + a_{1}x + a_0$, $a_n \nmid p$, then $f(x) \equiv 0 \pmod p$ at most $n$ solutions
\end{thm}
\begin{pf}
    induction, $(x - x_0)g(x) \equiv 0 \pmod{p}$, $\Z/p\Z$ no zero divisor
\end{pf}

\begin{thm}
    $p$ prime, $(\Z/p\Z)$ cyclic
\end{thm}
\begin{pf}
    $d \mid p - 1$, $S_d = \left\{ a : \text{order } d \right\}$, $x^d - 1 \equiv 0$ at most $d$ solution, then either 0 or $\phi(d)$ solution, but $\sum \phi(d) = p-1$
\end{pf}

\begin{itemize}
    \item primitive root
\end{itemize}

\begin{lemma}
    $p$ prime, then $\exists$ primitive root $g$ st $g^{p-1} = 1 + bp$ where $(b, p) = 1$
\end{lemma}
