\section{Queues}\label{sec:queues}

\begin{itemize}
    \item interarrival time $X_n$ with common distribution $F_X$
    \item service time $S_n$ with common distribution $F_S$
    \item $n$-th customer arrival time $T_n = \sum X_i$
    \item length of queue $Q(t)$
    \item $A/B/s$ ------ $F_X/F_S/\# $servers
\end{itemize}

\begin{example}\,
    \begin{itemize}
        \item $D(d)$ ------ deterministic
        \item $M(\lambda)$ ------ $Exp(\lambda)$ (Markovian)
        \item $\Gamma(\lambda, k)$
        \item $G$ ------ general
    \end{itemize}
\end{example}

\begin{example}\,
    \begin{itemize}
        \item $M/M/1$
        \item $M/D/1$
        \item $G/G/1$
    \end{itemize}
\end{example}

\begin{itemize}
    \item traffic intensity $\rho = \frac{\Ex(S)}{\Ex(X)}$
\end{itemize}

\subsection{M/M/1}\label{subsec:MM1}

\begin{setting}
    $M(\lambda)/M(\mu)/1$, $\lambda_n = \lambda$, $\mu_n = \mu$
\end{setting}

\begin{fact}
    $\rho = \frac{\lambda}{\mu}$
\end{fact}

\begin{thm}\,
    \begin{enumerate}
        \item if $\rho < 1$, then $\Pb(Q(t) = n) \rightarrow (1 - \rho)\rho^n = \pi_n$
        \item if $\rho \geq 1$, then $\Pb(Q(t) = n) \rightarrow 0$
    \end{enumerate}
\end{thm}

\begin{fact}
    can define underlying discrete random walk $Q_{n+1} =
    \begin{cases}
        Q_n + 1 & \text{with probability } \frac{\lambda}{\lambda + \mu} = \frac{\rho}{1 + \rho}\\
        Q_n - 1 & \text{with probability } \frac{\mu}{\lambda + \mu} = \frac{1}{1 + \rho}
    \end{cases}$ for $n \geq 1$, and $\Pb(Q_{n+1} = 1|Q_n = 0) = 1$
\end{fact}

\begin{fact}
    $Q_n$ is $\begin{cases}
                  \text{positive recurrent} & \text{if } \rho < 1\\
                  \text{null recurrent} & \text{if } \rho = 1\\
                  \text{transient} & \text{if } \rho > 1\\
    \end{cases}$
\end{fact}

\begin{itemize}
    \item waiting time of customer arrived at time $t$, $W$
\end{itemize}

\begin{thm}
    $\rho < 1$, queue in equilibrium, then $W \sim Exp(\mu - \lambda)$
\end{thm}

\begin{fact}
    expected queue length at equilibrium $ = \frac{\lambda}{\lambda + \mu}$
\end{fact}

\subsection{M/M/$\infty$}\label{subsec:m/m/}

\begin{setting}
    $\begin{cases}
         q_{i,i+1} = \lambda\\
         q_{i,i-1} = i\mu
    \end{cases}$
\end{setting}

\begin{thm}\,
    \begin{enumerate}
        \item $Q(t)$ positive recurrent
        \item invariant distribution $\pi \sim Poi(\rho)$
    \end{enumerate}
\end{thm}
\begin{pf}
    solve detail balanced for invariant, coupling to prove non-explosive
\end{pf}

\begin{setting}
    $M/M/1$ queue, $\rho < 1$
\end{setting}

\begin{itemize}
    \item $D_t$ ------ number of customers have departed queue up to time $t$
\end{itemize}

\begin{thm}[Burke's theorem]\,
    \begin{enumerate}
        \item At equalibrium, $D_t \sim Poi(\lambda)$
        \item $X_t$ independent from $(D_s : s \leq t)$
    \end{enumerate}
\end{thm}
\begin{pf}
    \begin{enumerate}
        \item fix $T$, time reversal, then Poisson process for all $T$, use independent increment criterion.
        \item $X_0$ independent to $[0,T]$, then reverse
    \end{enumerate}
\end{pf}

\subsection{Queues in tandem}\label{subsec:queues-in-tandem}

\begin{setting}
    two $M/M/1$ with $\lambda, \mu_1, \mu_{2}$
\end{setting}

\begin{thm}
    $X_t$, $Y_t$ queue length of first, second queue, then $(X, Y)$ positive recurrent Markov chain $\iff$ $\lambda < \mu_1, \mu_2$
    In this case, $\pi(m, n) = (1 - \rho_1)\rho_1^m (1 - \rho_2)\rho_n$, so $X_t, Y_t$ indepedent, geometric distributed
\end{thm}
\begin{pf}
    \begin{enumerate}
        \item (\textbf{Proof 1:}) $(m , n) \rightarrow \begin{cases}
                                                           (m + 1, n) & \text{with rate }\lambda\\
                                                           (m , n + 1) & \text{with rate }\mu_1 \text{ if } m \geq 1\\
                                                           (m , n - 1) & \text{with rate }\mu_2 \text{ if } n \geq 1\\
        \end{cases}$, then check directly.
        Rate bounded so non-explosive
        \item (\textbf{Proof 2:}) Burke's
    \end{enumerate}
\end{pf}

\begin{fact}
    r.v.\ independent while process not independent
\end{fact}

